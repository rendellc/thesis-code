\documentclass{article}


\usepackage[utf8]{inputenc}
\usepackage{graphicx}
\usepackage{xargs}
\usepackage[pdftex,dvipsnames]{xcolor}
\usepackage{import}
\usepackage{pdfpages}
\usepackage{biblatex}
\usepackage{amsmath}

\renewcommand{\tan}[1]{%
    \ensuremath{\text{tan}\left({#1}\right)}%
}
\renewcommand{\sin}[1]{%
    \ensuremath{\text{sin}\left({#1}\right)}%
}
\renewcommand{\cos}[1]{%
    \ensuremath{\text{cos}\left({#1}\right)}%
}
\renewcommand{\arctan}[1]{%
    \ensuremath{\text{arctan}\left({#1}\right)}%
}
\renewcommand{\vec}[1]{%
    \ensuremath{\mathbf{ {#1} }}%
}
\newcommand{\unitvector}[1]{%
    \ensuremath{\overrightarrow{e_{#1}} }%
}
\newcommand{\matrix}[1]{%
    \ensuremath{\mathbf{{#1}} }%
}
\newcommand{\homomatrix}[2]{%
    \ensuremath{%
      \begin{bmatrix}%
        {#1} & {#2} \\%
        \mathbf{0} & 1 %
      \end{bmatrix}%
    }
}



\usepackage{todonotes}
% https://tex.stackexchange.com/questions/9796/how-to-add-todo-notes
\newcommandx{\info}[2][1=]{%
    \todo[linecolor=OliveGreen,backgroundcolor=OliveGreen!25,bordercolor=OliveGreen,#1]{#2}%
}
\newcommandx{\improve}[2][1=]{%
    \todo[linecolor=Peach,backgroundcolor=Peach!25,bordercolor=Peach,#1]{#2}%
}


\usepackage{tabularx}
\newcolumntype{L}{>{\raggedright\arraybackslash}X}


\usepackage{cleveref}



\addbibresource{bibliography.bib}

\title{Project Thesis}
\author{Rendell Cale}
\date{August 2020}

\begin{document}

\maketitle

\import{draft/}{automotive-control-systems.tex}
\import{draft/}{on-developing-all-wheel-drive-model.tex}
\import{draft/}{wheel-dynamics.tex}
\import{draft/}{grip-phd.tex}
\import{draft/}{angular-momentum.tex}
\import{sections/}{rotations.tex}
\import{sections/}{rigid-body-kinematics.tex}
\import{sections/}{dynamics.tex}


\section{Outline}
\subsection{Modelling}
\subsubsection{Kinematics}
Describe the coordinate frames and kinematic equations.
\subsubsection{Dynamics}
Describe forces and torques acting on the system.
\paragraph{Friction}
\paragraph{Load transfer}
\subsubsection{Planar vehicle model}
Create model which neglect roll, pitch, and altitude. Maybe also assume that CG is aligned with wheels.
\subsection{Simulation}
\subsubsection{Challenges}
Load transfer makes friction force depend on acceleration, which mean we might have a differential algebraic equation.
\subsubsection{Solution}
Propose a simulation strategy given the challenges, and implement it (or find it in a package). 




\printbibliography

\end{document}
